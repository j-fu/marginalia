% Options for packages loaded elsewhere
\PassOptionsToPackage{unicode}{hyperref}
\PassOptionsToPackage{hyphens}{url}
%
\documentclass[
  10pt,
  ignorenonframetext,
]{beamer}
\usepackage{pgfpages}
\setbeamertemplate{caption}[numbered]
\setbeamertemplate{caption label separator}{: }
\setbeamercolor{caption name}{fg=normal text.fg}
\beamertemplatenavigationsymbolsempty
% Prevent slide breaks in the middle of a paragraph
\widowpenalties 1 10000
\raggedbottom
\setbeamertemplate{part page}{
  \centering
  \begin{beamercolorbox}[sep=16pt,center]{part title}
    \usebeamerfont{part title}\insertpart\par
  \end{beamercolorbox}
}
\setbeamertemplate{section page}{
  \centering
  \begin{beamercolorbox}[sep=12pt,center]{part title}
    \usebeamerfont{section title}\insertsection\par
  \end{beamercolorbox}
}
\setbeamertemplate{subsection page}{
  \centering
  \begin{beamercolorbox}[sep=8pt,center]{part title}
    \usebeamerfont{subsection title}\insertsubsection\par
  \end{beamercolorbox}
}
\AtBeginPart{
  \frame{\partpage}
}
\AtBeginSection{
  \ifbibliography
  \else
    \frame{\sectionpage}
  \fi
}
\AtBeginSubsection{
  \frame{\subsectionpage}
}
\usepackage{amsmath,amssymb}
\usepackage{iftex}
\ifPDFTeX
  \usepackage[T1]{fontenc}
  \usepackage[utf8]{inputenc}
  \usepackage{textcomp} % provide euro and other symbols
\else % if luatex or xetex
  \usepackage{unicode-math} % this also loads fontspec
  \defaultfontfeatures{Scale=MatchLowercase}
  \defaultfontfeatures[\rmfamily]{Ligatures=TeX,Scale=1}
\fi
\usepackage{lmodern}
\ifPDFTeX\else
  % xetex/luatex font selection
\fi
% Use upquote if available, for straight quotes in verbatim environments
\IfFileExists{upquote.sty}{\usepackage{upquote}}{}
\IfFileExists{microtype.sty}{% use microtype if available
  \usepackage[]{microtype}
  \UseMicrotypeSet[protrusion]{basicmath} % disable protrusion for tt fonts
}{}
\makeatletter
\@ifundefined{KOMAClassName}{% if non-KOMA class
  \IfFileExists{parskip.sty}{%
    \usepackage{parskip}
  }{% else
    \setlength{\parindent}{0pt}
    \setlength{\parskip}{6pt plus 2pt minus 1pt}}
}{% if KOMA class
  \KOMAoptions{parskip=half}}
\makeatother
\usepackage{xcolor}
\newif\ifbibliography
\setlength{\emergencystretch}{3em} % prevent overfull lines
\providecommand{\tightlist}{%
  \setlength{\itemsep}{0pt}\setlength{\parskip}{0pt}}
\setcounter{secnumdepth}{-\maxdimen} % remove section numbering
\ifLuaTeX
  \usepackage{selnolig}  % disable illegal ligatures
\fi
\IfFileExists{bookmark.sty}{\usepackage{bookmark}}{\usepackage{hyperref}}
\IfFileExists{xurl.sty}{\usepackage{xurl}}{} % add URL line breaks if available
\urlstyle{same}
\hypersetup{
  pdftitle={Some (not so) new command line utilities},
  pdfauthor={Jürgen Fuhrmann},
  hidelinks,
  pdfcreator={LaTeX via pandoc}}

\title{Some (not so) new command line utilities}
\author{Jürgen Fuhrmann}
\date{WIAS coffee lecture, 2023-04-28}

\begin{document}
\frame{\titlepage}

\begin{frame}
Where is my file?

What was the name of the file where I wrote that phrase?

How can I keep track of my git repository?

How do I convert my markdown text to LaTeX?

Stay up-to-date with tools written in Haskell, Rust an Go which
complement the classical UNIX command line canon.
\end{frame}

\begin{frame}{fzf}
\protect\hypertarget{fzf}{}
\begin{block}{aka ``Where is my file ?''}
\protect\hypertarget{aka-where-is-my-file}{}
\begin{itemize}
\tightlist
\item
  fast and powerful command-line fuzzy finder
\item
  helps you quickly search for files, directories and other items
\item
  supports integration with other command-line tools, such as bash, git,
  ag, and fd
\end{itemize}
\end{block}
\end{frame}

\begin{frame}[fragile]{fzf with bash completion}
\protect\hypertarget{fzf-with-bash-completion}{}
You want to run

\begin{verbatim}
$ command myfile
\end{verbatim}

but \texttt{myfile} is deep in your directory hierarchy.

\begin{itemize}
\tightlist
\item
  enter \texttt{command} and \texttt{**\textless{}TAB\textgreater{}}
\item
  choose file once found, press \texttt{enter}
\item
  this will run
\end{itemize}

\begin{verbatim}
$ command somewhere/deep/myfile
\end{verbatim}
\end{frame}

\begin{frame}{rg}
\protect\hypertarget{rg}{}
aka ``What was the name of the file where I wrote that phrase?''

\begin{itemize}
\tightlist
\item
  search tool for the command line, alternative to the traditional grep
  command.
\item
  regular expressions and is optimized for speed, making it ideal for
  searching large files and directories.
\item
  support for search and replace operations, file type filtering
\end{itemize}
\end{frame}

\begin{frame}[fragile]{rg usage}
\protect\hypertarget{rg-usage}{}
\begin{verbatim}
$ find . -name "*.md" | xargs grep myphrase
\end{verbatim}

vs

\begin{verbatim}
$ rg -tmd myphrase
\end{verbatim}

\ldots{} and it is much faster
\end{frame}

\begin{frame}{gitui}
\protect\hypertarget{gitui}{}
\begin{itemize}
\tightlist
\item
  terminal-based graphical user interface for Git
\item
  easy-to-use interface for common Git operations like commit, push, and
  pull
\item
  advanced features like branch management and stash management
\end{itemize}
\end{frame}

\begin{frame}{pandoc}
\protect\hypertarget{pandoc}{}
\begin{itemize}
\tightlist
\item
  command-line utility for converting files from one format to another,
  such as Markdown to HTML, LaTeX to Word, or EPUB to PDF.
\item
  can also be used as a library in other programs or scripts to automate
  document conversion tasks
\end{itemize}
\end{frame}

\begin{frame}{Acknowledgement}
\protect\hypertarget{acknowledgement}{}
\begin{itemize}
\tightlist
\item
  ChatGPT for creating intial slide texts
\item
  Discussion on this is intended for the next coffee lecture
\end{itemize}
\end{frame}

\end{document}
